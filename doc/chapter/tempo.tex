\section{\ttfamily TempoExtractor}
\emph{TempoExtractor} is the abstract superclass of which all tempo extractors
should be derived.

\subsection{\ttfamily InterOnsetTempoExtractor}
As the name suggests, \emph{InterOnsetTempoExtractor} performs the Inter Onset
Intervals as described in the lecture slides 5.10 through 5.14. The method
calculates the most common gap size between the onsets inside a set of onsets. 

Because of the inaccuracy that evolves when adding vague \texttt{double}
values and the necessarity to provide some sort of categorization and
clusterization an indirectional mapping from the \texttt{double} value of the
gap size to the \texttt{int} value of the number of occurencies was established.
In order to reach this goal actually two mappings are done:

\begin{enumerate}
  \item \texttt{double} $\rightarrow$ \texttt{int}: specified by the constant
  \texttt{TEMPO\_KEY\_TOLERANCE}, given distances are separated into different
  categories which are consecutively numbered using the \texttt{int} values. To
  be more precise, if \texttt{distance $\pm$ TEMPO\_KEY\_TOLERANCE}
  meets one key in the map, it is put into the same category. Otherwise it is
  put into a new one.
  \item \texttt{int} $\rightarrow$ \texttt{int}: The \emph{value} of the
  mapping in 1 is the \emph{key} of the second mapping, which provides the
  number of occurencies.
\end{enumerate}

Of course this is to be done if the distance is within the range of usual tempi.
Therefore distances that are smaller than \texttt{MIN\_TEMPO} or greater than
\texttt{MAX\_TEMPO} are not taken into account.

The tempo then is returned in \emph{beats per minute} (bpm).
