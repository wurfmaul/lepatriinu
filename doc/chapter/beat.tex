\section{\ttfamily BeatDetector}
Also \emph{BeatDetector} is an abstract class, thats purpose is to be the
superclass of all beat detection algorithms. 

\subsection{\ttfamily AutoCorrelationBeatDetector}
This subclass implements beat detection based on auto correlation. It
accommodates an enumeration, named \emph{Mode}, which allows to switch between
three different ideas and implementations:

\begin{enumerate}
  \item \texttt{AUTO\_TEMPO\_CORRELATION} simply implements the idea of the
  pulse train, described on slides 5.15 through 5.19. To tackle the inaccuracy the
  list of onsets will always carry along we use a threshold, called
  \texttt{AUTO\_PHASE\_TOLERANCE} which, after excessive testing we found best
  to be set around 0.21 and pushed it up to 48\% (average over all 18 provided data
  sets).
  \item \texttt{AUTO\_ONSET\_CORRELATION} represents the implementation of the
  IOI idea presented on lecture slides 5.10 through 5.14.
  \item \texttt{AUTO\_SPECTRAL\_CORRELATION} was the attempt to improve the
  onset correlation by trying to reinterprete the formulas as to be the spectral data
  instead of the onsets. It didn't work out that well (less than 4\% over 18
  data sets).
\end{enumerate}

\subsection{Best results}
As onset correlation did a rather poor job and spectral correlation an even
worse, we stuck with tempo correlation as the algorithm of choice.
