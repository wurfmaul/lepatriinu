\subsection{\ttfamily Analyzer}
The \emph{Analyzer} class defines all the constants that are needed for any
algorithm in the project to succeed. In order to make central configurations
possible, the constants are collected in this single class.

Furthermore it provides the central interface to the
\texttt{at.cp.jku.teaching.amprocessing} project. It is initialized with a
pre-processed (e.g. FFT) audio file of type \texttt{Audiofile}. The order of
usage is important. In the first place, onset detection can be done. This
information can directly be retrieved from the audio file. The found onsets are
the base for tempo extraction. Therefore the onset list has to be provided as
parameter of the tempo extraction function. Finally beat detection can be
performed. In order to be able to use the best algorithms both the onset list as
well as the calculated tempo should be provided.

\begin{enumerate}
  \item Onset detection: \texttt{onsets = performOnsetDetection()}
  \item Tempo extraction: \texttt{tempo = performTempoExtraction(onsets)}
  \item Beat Detection: \texttt{beats = performBeatDetection(onsets, tempo)}
\end{enumerate}