\chapter{Architecture} \label{cpt:architecture}

\section{Class structure}
The project was separated into several classes. The given framework simple calls
methods which are defined in the package \texttt{at.jku.amp.lepatriinu}.

\begin{itemize}
  \ttfamily
  \item at.jku.amp.lepatriinu.Analyzer
  \item at.jku.amp.lepatriinu.BeatDetector
  \begin{itemize}
    \item at.jku.amp.lepatriinu.AutoCorrelationBeatDetector
  \end{itemize}
  \item at.jku.amp.lepatriinu.OnsetDetector
  \begin{itemize}
    \item at.jku.amp.lepatriinu.GroundTruthPicker 
    \item at.jku.amp.lepatriinu.HighFrequencyOnsetDetector
    \item at.jku.amp.lepatriinu.SimpleOnsetDetector
    \item at.jku.amp.lepatriinu.SpectralFluxOnsetDetector
  \end{itemize}
  \item at.jku.amp.lepatriinu.TempoExtractor
  \begin{itemize}
    \item at.jku.amp.lepatriinu.InterOnsetTempoExtractor
  \end{itemize} 
\end{itemize}

\subsection{\ttfamily Analyzer}
The \emph{Analyzer} class defines all the constants that are needed for any
algorithm in the project to succeed. In order to make central configurations
possible, the constants are collected in this single class.

Furthermore it provides the central interface to the
\texttt{at.cp.jku.teaching.amprocessing} project. It is initialized with a
pre-processed (e.g. FFT) audio file of type \texttt{Audiofile}. The order of
usage is important. In the first place, onset detection can be done. This
information can directly be retrieved from the audio file. The found onsets are
the base for tempo extraction. Therefore the onset list has to be provided as
parameter of the tempo extraction function. Finally beat detection can be
performed. In order to be able to use the best algorithms both the onset list as
well as the calculated tempo should be provided.

\begin{enumerate}
  \item Onset detection: \texttt{onsets = performOnsetDetection()}
  \item Tempo extraction: \texttt{tempo = performTempoExtraction(onsets)}
  \item Beat Detection: \texttt{beats = performBeatDetection(onsets, tempo)}
\end{enumerate}

\subsubsection{Constants}
In the following section the used constants are explained.

\begin{table}[htp]
	\begin{tabular}{|l|l|l|l|}
		\hline
			\bf Constant & \bf Type & \bf Default value & \bf Description\\
		\hline \hline
			ONSET\_DETECTOR & OnsetDetector & OnsetDetector.FLUX & specifies the used
			onset detection algorithm.\\
		\hline
			 &  &  & \\
		\hline
			 &  &  & \\
		\hline
			 &  &  & \\
		\hline
			 &  &  & \\
		\hline
			 &  &  & \\
		\hline
			 &  &  & \\
		\hline
			 &  &  & \\
		\hline
			 &  &  & \\
		\hline
			 &  &  & \\
		\hline
			 &  &  & \\
		\hline
			 &  &  & \\
		\hline
			 &  &  & \\
		\hline
		\hline
	\end{tabular}
	\caption{Constants in \texttt{Analyzer}}
	\label{tbl:constants}
\end{table}

% 	  = ;
% 	private static final TempoExtractor TEMPO_EXTRACTOR = TempoExtractor.IOTE;
% 	private static final BeatDetector BEAT_DETECTOR = BeatDetector.AUTO;
% 	
% 	// GENERAL CONSTANTS
% 	public static final boolean DEBUG_MODE = true;
% 
% 	// ONSET DETECTION CONSTANTS
% 	public static final int THRESHOLD = 13;
% 	public static final int THRESHOLD_RANGE = 5;
% 	public static final boolean FLUX_USE_TOTAL_ENERGY = true;
% 	public static final boolean HIFQ_USE_WPHACK = true;
% 	public static final boolean USE_MOUNTAIN_PEAKPICK = true;
% 	public static final boolean PEAKPICK_USE_MEAN = false;
% 
% 	// BEAT DETECTOR CONSTANTS
% 	public static final boolean AUTO_USE_ONSETS = false;
% 	public static final Mode AUTO_FUNCTION = Mode.AUTO_TEMPO_CORRELATION;
% 	public static final double AUTO_PHASE_TOLERANCE = 0.21;
% 
% 	// TEMPO EXTRACTOR CONSTANTS
% 	public static final double MIN_TEMPO = 0.3; // = 200 bpm
% 	public static final double MAX_TEMPO = 1; // = 60 bpm
% 	public static final double TEMPO_KEY_TOLERANCE = 0.0195;

\section{Graphical user interface}