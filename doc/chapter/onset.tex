\section{\ttfamily OnsetDetector}
\emph{OnsetDetector} is the abstract superclass of all onset detection
algorithm's classes. It provides two different peak picking methods which can be
used by any sub class.

\begin{enumerate}
  \item The first peak picking method is taken from lecture slides (5.40) and
  called \emph{Adaptive Thresholding}.
  \item The second one is a slight modification of the adaptive thresholding
  algorithm. Instead of a threshold it simply isolates the highest magnitudes by
  zeroing everything but the peak as well as close neighbors that are lower. As
  a result a clear list of onsets remains. We called this method \emph{mountain
  climbing}.
\end{enumerate}

\paragraph{Constants}
\begin{itemize}
  \item \texttt{USE\_MOUNTAIN\_PEAKPICK} defines which peakpicking method to
  use. If constant is \texttt{true}, \emph{mountain peakpicking} is chosen.
  Otherwise \emph{adaptive thresholding} is performed.
  \item \texttt{THRESHOLD\_RANGE} is used to specify the \texttt{int} size of
  the shifted slice used in \emph{adaptive thresholding} as well as the range of
  zoroed neighbors in \emph{mountain climbing}. The best results were produced
  by a range of \texttt{5}.
  \item \texttt{PEAKPICK\_USE\_MEAN} defines whether to use mean
  (\texttt{true}) or median (\texttt{false}) in order to select the threshold
  that is later on applied to all the onset calculations of \emph{adaptive
  thresholding}.
  \item \texttt{THRESHOLD} is used to define a fixed \texttt{int} threshold for
  choosing the magnitude peaks when using \emph{mountain climbing}. The value
  that prouced the best results in our experiments was \texttt{13}. 
\end{itemize}

\subsection{\ttfamily SimpleOnsetDetector}

\subsection{\ttfamily SpectralFluxOnsetDetector}

\subsection{\ttfamily HighFrequencyOnsetDetector}

\subsection{\ttfamily GroundTruthPicker}